% !TEX encoding = UTF-8 Unicode
\documentclass[
10pt,
aspectratio=169
]{beamer}

\usetheme{LSU}
\usefonttheme{professionalfonts}
\setbeamercovered{transparent=10}

% ==================== FONTS ====================
\usepackage{fontspec}
\setmainfont{Times New Roman}
\setsansfont{TeX Gyre Heros}
\setmonofont{Inconsolata}

% ==================== CORE PACKAGES ====================
\usepackage{tikz}
\usetikzlibrary{fadings}

\usepackage{listings}
\usepackage{amsmath}
\usepackage{multirow}
\usepackage{booktabs}
\usepackage{threeparttable}
\usepackage{bm}
\usepackage[ruled]{algorithm2e}

\setbeamertemplate{caption}[numbered]

% ==================== TOC STYLE ====================
\setbeamertemplate{section in toc}
{\hspace*{1em}\inserttocsectionnumber.~\inserttocsection\par}

\setbeamertemplate{subsection in toc}
{\hspace*{2em}\inserttocsectionnumber.\inserttocsubsectionnumber.~\inserttocsubsection\par}

\setbeamerfont{subsection in toc}{size=\small}

% % ==================== SECTION PAGE ====================
% \AtBeginSection[]{
% 	\begin{frame}
% 		\frametitle{Contents}
% 		\tableofcontents[currentsection,hideallsubsections]
% 	\end{frame}
% }

% ==================== MACROS ====================

\newcommand{\tabincell}[2]{\begin{tabular}{@{}#1@{}}#2\end{tabular}}


\makeatletter
\newcommand{\rmnum}[1]{\romannumeral #1}
\newcommand{\Rmnum}[1]{\expandafter\@slowromancap\romannumeral #1@}
\makeatother

\newcommand{\upcite}[1]{\textsuperscript{\cite{#1}}}

% ==================== TITLE INFO ====================
\title{YOUR TITLE}
\subtitle{SUBTITLE}
\author[Huakang Lin]{Huakang Lin}
\email{hlin29@lsu.edu}
\institute[LSU]{Computer Science and Engineering\\Louisiana State University}
\date{\today}

\begin{document}

% ==================== TITLE PAGE ====================
\maketitle

%==============================
\section{Introduction}
%==============================

\begin{frame}{Background}
\begin{itemize}
    \item Modern cloud platforms process massive and heterogeneous job requests.
    \item Resource availability in these systems is \textbf{highly dynamic}.
    \item Inaccurate resource estimation can lead to:
    \begin{itemize}
        \item Over-provisioning (resource waste)
        \item Under-provisioning (job failures)
    \end{itemize}
    \item This motivates the study of \textbf{uncertainty-aware scheduling}.
\end{itemize}
\end{frame}


%==============================
\section{Related Work}
%==============================

\begin{frame}{Related Work}

\begin{itemize}
    \item Predictive scheduling has been widely studied in cloud computing systems \cite{Hangdong2020}.
    \item Recent work explores learning-based resource allocation strategies \cite{Mo2019}.
    \item However, most existing methods:
    \begin{itemize}
        \item Assume a fixed resource distribution
        \item Ignore workload volatility
        \item Do not consider user experience
    \end{itemize}
\end{itemize}

\end{frame}


%==============================
\section{Motivation}
%==============================

\begin{frame}{Motivation}

\begin{columns}
\column{0.6\textwidth}
\begin{itemize}
    \item Workloads in practice are highly unpredictable
    \item Cloud capacity follows irregular patterns
    \item Static scheduling policies fail under uncertainty
    \item We aim to design a:
    \begin{itemize}
        \item Robust
        \item Adaptive
        \item User-aware
    \end{itemize}
    scheduling mechanism
\end{itemize}

\column{0.4\textwidth}
\begin{figure}
    \centering
    \includegraphics[width=\linewidth]{example-image-a}
    \caption{Dynamic resource demand}
\end{figure}
\end{columns}

\end{frame}


%==============================
\section{Problem Formulation}
%==============================

\begin{frame}{Problem Formulation}

The scheduling problem is formulated as:

\[
\max_{\mathbf{x}} \quad \sum_{i=1}^{N} v_i x_i
\]

subject to:

\[
\sum_{i=1}^{N} r_i x_i \leq \hat{C} + \delta
\]

where

\begin{itemize}
    \item $x_i \in \{0,1\}$ — whether job $i$ is selected
    \item $v_i$ — priority or value
    \item $r_i$ — resource requirement
    \item $\hat{C}$ — predicted available capacity
    \item $\delta$ — uncertainty margin
\end{itemize}

This is an instance of \textbf{stochastic resource-constrained optimization}.

\end{frame}


%==============================
\section{Algorithm Design}
%==============================

\begin{frame}{Proposed Method}

\textbf{Main components:}

\begin{enumerate}
    \item Probabilistic resource estimation
    \item Risk-aware job admission control
    \item Priority-based selection strategy
\end{enumerate}

\bigskip

\textbf{Objective:}

\begin{itemize}
    \item Maximize total utility
    \item Minimize resource violation risk
    \item Improve fairness among users
\end{itemize}

\end{frame}


%==============================
\section{Experimental Setup}
%==============================

\begin{frame}{Experimental Setup}

\begin{itemize}
    \item Platform: Simulated cloud cluster
    \item Jobs: Synthetic + real workload traces
    \item Metrics:
    \begin{itemize}
        \item Resource utilization
        \item Throughput
        \item Violation rate
    \end{itemize}
    \item Compared against:
    \begin{itemize}
        \item Greedy scheduling
        \item Random selection
    \end{itemize}
\end{itemize}

\end{frame}


%==============================
\section{Conclusion}
%==============================

\begin{frame}{Conclusion}

\begin{itemize}
    \item We study scheduling under uncertain resource availability.
    \item We design an uncertainty-aware scheduling framework.
    \item Experiments show:
    \begin{itemize}
        \item Higher utilization
        \item Lower violation rates
        \item Better fairness
    \end{itemize}
    \item Future work:
    \begin{itemize}
        \item Multi-resource scheduling
        \item Real-world deployment
    \end{itemize}
\end{itemize}

\end{frame}


%==============================
\section{References}
%==============================

\begin{frame}[allowframebreaks]{References}
\footnotesize
\bibliographystyle{IEEEtran}
\bibliography{References}
\end{frame}


% ==================== THANK YOU ====================

\LSUThankYou{Huakang Lin}{hlin29@lsu.edu}

\end{document}
